\usepackage[12pt]{article}
\begin{document}

\section{Dabatase Management System (DBMS) Fundamentals}
	\begin{itemize}
		\item self describing data
		\item program-data separation (independence in book)
		\item data abstraction
		\item network/multiuser access (transaction control)
		\item client/server architecture
	\end{itemize}

	\section{History- Early Models}
	\item pre-1960s
	\item 1962: term "database"
	\item 1964: GE's IDS by Charles Bachman (Turing Award 1973)
		\begin{itemize}
			\item network model
				\begin{itemize}
					\item records	record-type
					\item sets		set-type
				\end{itemize}
		- CODASYL- standard in 1969
		\end{itemize}
	\item 1966-68: IBM's IMS (Information Management System for Apollo); first hierarchial model DB
	\item navigational model: both IDS and IMS
	\item 1970: relational model (IBM's E.F. Codd)
	\item 1977: System R (IBM)- SEQUEL -> SQL
	\item 1977-79: Oracle
	\item 1973: INGRES- Berkeley- Michael Stonebraker \& Eugene Wong
	\item 1985: Postgres
													8/31/15
	1960: navigational databases
	1970: relational model
	NoSQL
		- scalability
		- flexibility (JSON, XML)
		- fault tolerance
Relational Databases
	- Rows are called tuples
	- A relation is a set of tuples
	-** Tuple: collection of named/ordered values ('CPW', 'CSCI 403', 'Database Management', 3)
		- Attribute names: 		    (INSTRUCTOR, COURSE ID, TITLE, HOURS)
		- Domain: dom(INSTRUCTOR) = 'string'
			- may include NULL
	-** Relation schema:
		- R = R(A1, A2, ..., An)
		- Set of attributes to store in relation
		- R has degree n
		- Ai has domain dom(Ai)
	-** Relation (a.k.a. relation state):
		- A set of tuples consistent with relation schema
		-r(R) is a subset of (dom(A1) * dom(A2) * ... * dom(An)

	Note: "relation is a set" -> no ordering of tuple in relation
		also -> no duplicates (* in practice)
	Note: "attributes are set" ->ordering of sttributes in a relation doesn't matter

\end{document}
